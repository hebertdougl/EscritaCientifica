\begin{resumo}

A qualidade interna é o principal fator de sucesso de projetos de software, pois corresponde a aspectos primordiais do software tais como manutenibilidade e segurança. Softwares com boa qualidade interna proporcionam maior produtividade uma vez que possibilitam a criação de mais testes automatizados, são mais compreensíveis, reduzem o risco de \emph{bugs} e facilitam as modificações e evoluções no código. Portanto, o Engenheiro de Software é um dos principais responsáveis por este sucesso uma vez que deve reunir um conjunto de habilidades e conhecimentos que o permitam aplicar práticas, técnicas e ferramentas para a criação de softwares seguros e com bom \emph{design}. Diante disso, este trabalho de conclusão de curso aborda as principais ideias e conceitos relacionados à melhoria contínua do código-fonte. Neste sentido, nesta monografia é destacada a importância da realização de atividades contínuas relacionadas ao \emph{design} e segurança ao longo de todo o projeto de software, além de discutir a importância da utilização de métricas estáticas de código-fonte para suportar a tomada de decisões, tanto a nível técnico quanto gerencial. Neste sentido, é apresentado o conceito de Cenários de Decisões que definem uma abstração para escolha e interpretação de métricas, além da proposta de exemplos de utilização destes Cenários para medição da segurança de software. Para suportar a utilização de cenários e métricas no desenvolvimento de software, este trabalho ainda contempla a evolução da plataforma livre de monitoramento de código-fonte chamada Mezuro e a construção de uma solução de DataWarehousing. Por fim, serão definidos protocolos para realização de estudos de casos que permitam avaliar estas ferramentas e compreender a correlação entre a qualidade interna e vulnerabilidades de software, através de métricas.

 \vspace{\onelineskip}
    
 \noindent
 \textbf{Palavras-chaves}: Métricas; Design; Segurança; Monitoramento; DataWarehousing; Cenários de Decisões;
\end{resumo}

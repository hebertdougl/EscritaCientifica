\chapter*[Introdução]{Introdução}
\addcontentsline{toc}{chapter}{Introdução}

A Engenharia de Software tem evoluídos seus métodos e técnicas para prover melhorias no desenvolvimento de software com objetivos baseados em cumprimento de prazos e orçamentos assim como a implementação de produtos que atendem parâmetros de qualidades desejados. Estas melhorias são observáveis em diferentes pontos, desde o processo ao produto, cujos objetivos e prioridades podem variar de acordo com a metodologia de desenvolvimento. Apesar de suas diferenças conceituais e de valores, a maior parte dos métodos preveêm processos e técnicas referentes ao design, testes e medição, que visam garantir a qualidade do software em desenvolvimento.

%

No contexto de projetos que adotam metodologias ágeis observa-se que tanto a qualidade interna quanto a qualidade externa do software são preponderantes, pois são fatores fundamentais para suportar a simplicidade, o feedback contínuo e adaptação à mudanças, valores que solidificam o desenvolvimento ágil. A qualidade interna do software é observada a partir de atributos de qualidades na perspectiva de desenvolvimento que, segundo Berander (\citeyear{berander2005}), se resumem em corretude, testabilidade, flexibilidade, portabilidade, reusabilidade, interoperabilidade, analisabilidade, adaptatividade e estabilidade. As práticas ressaltadas pela metologia Extreme Programming \cite{beck2000} visam realçar os valores dos atributos destacados. O design simples pode ser obtido através de técnicas como o Desenvolvimento Orientado à Testes (CITAR UMA REFERÊNCIA) e o Refactoring (CITAR OUTRA REFERÊNCIA), que por sua vez influenciam diretamente os atributos testabilidade, reusabilidade e adaptatividade. Ambas as técnicas se baseiam fortemente em testes unitários (CITAR OUTRA REFERÊNCIA) que provê a segurança necessária para realização de mudanças assim como o feedback automatizado da manutenção do software. O Pair Programming (CITAR REFERÊNCIA), dentre outras técnicas, também possui papel fundamental na garantia da qualidade interna, uma vez que exercita a programação e revisão ao mesmo tempo, reduzindo a ocorrência de não-conformidades técnicas e inserção de bugs. Por outro lado, a qualidade externa do software pode ser alcançada a partir do envolvimento do cliente ao longo das atividades de desenvolvimento e, principalmente, a partir de entregas contínuas de software com valor de negócio.

%

Valores semelhantes podem ser observados nas comunidades de desenvolvimento de softwares livres refletindo diretamente na alta qualidade do código produzido em diversos projetos livres (\cite{schmidt2001}; \cite{halloran2002}; \cite{michlmayr2000}). Essas comunidades adotam a padronização de código e testes automatizados para manter a qualidade interna do código e incentivar a contribuição de diversos desenvolvedores.

%
A melhoria da qualidade interna do código apoia a melhoria contínua do processo oferecendo subsídios para que a equipe de desenvolvimento aumente sua produtividade e implemente novas funcionalidades com maior facilidade. Beck (\citeyear{beck2007}) corrobora esta afirmação ao destacar que a maior parte do tempo utilizado por um Programador ao inserir novas funcionalidades é destinado ao entendimento do código em manutenção. Neste sentido, a medição pode ser utilizada como um processo de apoio ao acompanhamento destas melhorias através do estabelecimento de metas e indicadores que indiquem oportunidades de melhorias observáveis do produto. Em um cenário otimista, os próprios Engenheiros de Software podem adotar como prática a medição do código-fonte para auxiliar as tomadas de decisões, ou até mesmo para avaliação do código inserido ou da aplicação de refactoring.
%

Portanto, neste trabalho serão exploradas a utilização de métricas de monitoramento de código-fonte para compreender e estabelecer possíveis relações existentes entre as mesmas. Assim, espera-se identificar as oportunidades de utilização de métricas na melhoria contínua do processo e, consequentemente, na qualidade interna do produto a partir do estabelecimento de cenários, compostos a partir da análise de correlação de métricas, que evidenciem as boas e más características do design de um projeto com o objetivo de facilitar a interpretação e evitar possíveis equívocos que são baseados em análises errôneas sobre métricas isoladas, sobre correlações inexistentes ou até mesmo a escolha de métricas inadequadas cujos problemas são discutidos em \cite{chidamber1994}.

%------------------------------------------------------------------------------%

\section{Objetivos}

O objetivo deste trabalho permeia um estudo teórico de conceitos relacionados a métricas de monitoramento de código-fonte, buscando identificar correlações existentes entre algumas métricas e suas interpretações que evidenciem tanto características de bom design, tais como Código Limpo \cite{almeida1994}, quanto bad smells (CITAR REFERÊNCIA). O estabelecimento destes cenários permite a seleção de métricas adequadas baseadas em determinados objetivos de melhorias, facilta a interpretação tanto a nível gerencial quanto a nível de desenvolvimento e apoia a melhoria contínua de software a partir da adoção do monitoramento de código-fonte como uma prática de desenvolvimento.

%

Além disso, tem-se como objetivo nesta monografia evoluir a plataforma livre de monitoramento de código-fonte Mezuro para suportar a criação de configurações que caracterizem os cenários estabelecidos. Assim, pretende-se evidenciar como estes cenários podem ser utilizados através da ferramenta de análise estática de código automatizada incorporados as boas práticas de desenvolvimento das Engenharia de Software.

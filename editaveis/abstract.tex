\begin{resumo}[Abstract]
  \begin{otherlanguage*}{english} 
  
The internal quality is the key success factor of software projects because it corresponds to the main aspects of the software such as maintainability and security. Software with good internal quality provides more productivity since it supports the creation of more automated tests, are more understandable, reduce the risk of bugs and make the code changes and developments easier to be done. Therefore, the Software Engineer is a major contributor for this success since he must gather a set of skills and knowledge in order to apply practices, techniques and tools for creating secure and well design software. Thus, this completing of course work covers the main ideas and concepts related to continuous improvement of source code. In this sense, in this monograph is highlighted the importance of conducting ongoing activities related to design and security throughout the software project, and discuss the importance of using static source code metrics to support decision making at managerial level and technical as well. In this sense, we present the concept of Decisions Scenarios that define an abstraction for metrics' choice and interpretation, as well as proposals of examples in order to use scenarios for measuring software security. To support the use of scenarios and metrics in software development, this work also includes the evolution of a source code monitoring free software called Mezuro and building a datawarehousing solution. Finally, will be defined protocols for conducting case studies to evaluate these tools and understand the correlation between internal quality and software vulnerabilities through metrics.
  
  \vspace{\onelineskip}
 
  \noindent 
  \textbf{Key-words}: Metrics; Design; Security; Monitoring; DataWarehousing; Decisions Scenarios;
  \end{otherlanguage*}
\end{resumo}



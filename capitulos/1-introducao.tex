\chapter{Introdução}
\label{cap-introducao}

%------------------------------------------------------------------------------%
\section{Tema}
Padrões para implentação de mudanças em software livre.
\section{Justificativa}

A engenharia de software nos últimos anos sofreu uma grande evolução em vários pontos, desde o amadurecimento de vários processos ao emprego de métodos e técnicas com objetivo de desenvolver software cumprindo prazos e orçamentos desejados.

A manutenção de software é uma das atividades mais criticas no ciclo de vida de um software pois a partir do momento em que um software está em um ambiente de produção, principalmente em sistemas legados, vários \emph{bugs}, defeitos e prospostas para melhoria são detectados. Gerenciar diversas requisições de mudança afim de se analisar, priorizar, implementar, verificar e validar é um processo complexo que demanda muito tempo e experiência da equipe.

Os custos envolvidos na atividade de manutenção torna o tema relevante, na dissertação de mestrado de \citeonline{paduelli2007manutencao} ele evidencia através de revisão sistemática realizada que a manutenção consome a maior parte dos custos envolvidos no ciclo de vida de um software, segundo \citeonline{pressman2011engenharia} verifica-se que o custo na atividade de manutenção em sua totatilidade supera 50\% do investimento total em um software. 

Realizar trabalhos relacionadas a atividade de manutenção de software são importantes pois agregam valor a engenharia de software, já que este ponto ainda carece de um detalhamento mais específico afim de equacionar os problemas que ocorrem em sua execução.

\section{Problema}

Segundo a literatura mesmo com a evolução da engenharia de software, atualmente problemas na atividade de manutenção de software são recorrentes e equivalentes aos que foram relatados por autores no passado. O que demonstra que houve pouca evolução nesta atividade ou os problemas são complexos para serem mitigados. Alguns problemas indicados por \citeonline{lientz1980software} são: 
\begin{itemize}
\item \textbf Baixa qualidade da documentação dos sistemas;
\item \textbf Necessidades dos usuários por constantes melhorias e novas funcionalidades;
\item \textbf Falta de uma equipe de manutenção
\item \textbf Falta de comprometimento com cronogramas;
\item \textbf Treinamento inadequado da equipe de manutenção e rotatividade de profissionais.
\end{itemize}

A estes problemas pode-se dizer que estão diretamente relacionados aos altos custos pois muitos deles são gerenciais ou exigem um grande esforço por parte da equipe a partir disso torna-se imprescindível investigar formas para os minimizar afim de reduzi-los ao máximo

\subsection{Questões de Pesquisa}

%
Sabendo dos problemas encontrados no contexto de manutenção de software, este artigo busca responder as seguintes questões de pesquisa:

%

\begin{itemize}
\item \textbf{QP1} Como implementar mudança em projetos de software livre de modo que os problemas sejam minimizados?
\end{itemize}

\section{Objetivos}

\subsection{Objetivo Geral}

O objetivo deste trabalho consiste no estudo teórico sobre conceitos relacionadas a manutenção de software livre buscando propor padrões no contexto da implementação em mudanças no projeto.

\subsection{Objetivo Específico}

Para responder estas questões de pesquisa, busca-se atingir os seguintes objetivos específicos:
\begin{itemize}
\item \textbf Verificar como funciona o mecanismo para requisições de mudança em software livre.
\item \textbf Identificar quais os problemas principais em manutenção de software livre.
\item \textbf Priorizar quais as dificulades que causam mais impacto para desenvolvimento.
\item \textbf Agrupar problemas com características semelhantea.
\item \textbf Propor padrões para requisição de mudança.
\end{itemize}


%------------------------------------------------------------------------------%

\section{Metodologia e Pesquisa}

%
Para alcançar o objetivos objetivos propostos no trabalho e responder as perguntas de pesquisa levantadas, serão realizados estudos teóricos através de revisão bibliográfica para compreensão e definição dos conceitos de manutenção de software, software livre e padrões de software.

Além das definições obtidas a partir da revisão bibliográfica, serão aplicados questionários para verificar se dos problemas listados na literatura quais são os mais recorrentes nas comunidades de software livre. Neste sentido baseado na literatura, espera-se agrupar estes problemas afim de se encontrar uma caracterśitica semelhante entre os mesmos.

E por fim, baseado na revisão bibliográfica procura-se elaborar ao menos um padrão de manutenção que auxilie as empresas na atvidade de manutenção de software.


%------------------------------------------------------------------------------%
